\documentclass{sig-alternate}
\usepackage{graphicx}
\usepackage{listings}
\def\lstxml{
  \lstset{language=XML,
    keywordstyle=\ttfamily,
    identifierstyle=\ttfamily\bfseries, 
    % commentstyle=\color{Brown},
    stringstyle=\ttfamily,
    showstringspaces=false,
    columns=[l]flexible, %% , basewidth={0.5em,0.4em}
    morekeywords={encoding,
      mrow,math,mfrac,mi,msqrt,mo,mn,span,nobr,img}
  }
}

\def\lstjs{
  \lstset{language=Java,
    keywordstyle=\ttfamily,
    identifierstyle=\ttfamily\bfseries, 
    % commentstyle=\color{Brown},
    stringstyle=\ttfamily,
    showstringspaces=false,
    columns=[l]flexible, %% , basewidth={0.5em,0.4em}
    morekeywords={cvox,Api,Math,defineRule}
  }
}

\def\lsttex{
  \lstset{language={[LaTeX]TeX},
    texcsstyle=*\bfseries,
    keywordstyle=\ttfamily,
    identifierstyle=\ttfamily\bfseries, 
    % commentstyle=\color{Brown},
    stringstyle=\ttfamily,
    showstringspaces=false,
    columns=[l]flexible, %% , basewidth={0.5em,0.4em}
    moretexcs=intertext,
    morekeywords={sum,Sigma}
  }
}

\newcommand\ednote[1]{\typeout{There is still a note!!!}%
  {\bf EDNOTE: #1}}

\newcommand\edbf[1]{\typeout{There is still an editor's note!!!}%
  \textbf{EDNOTE: #1}}


\begin{document}
\title{Accessibility to Scientific Material:\\ The Case of Speaking Math}
\numberofauthors{2} %  in this sample file, there are a *total*

\author{
% You can go ahead and credit any number of authors here,
% e.g. one 'row of three' or two rows (consisting of one row of three
% and a second row of one, two or three).
%
% The command \alignauthor (no curly braces needed) should
% precede each author name, affiliation/snail-mail address and
% e-mail address. Additionally, tag each line of
% affiliation/address with \affaddr, and tag the
% e-mail address with \email.
%
% 1st. author
\alignauthor
Volker Sorge\titlenote{Work was done while author spent a sabbatical at Google,
  Inc., Mountain View, CA, USA.}\\
       \affaddr{School of Computer Science}\\
       \affaddr{The University of Birmingham, UK}\\
       \email{V.Sorge@cs.bham.ac.uk}
% 2nd. author
\alignauthor
Charles Chen, T.V. Raman, David Tseng\\
       \affaddr{Google, Inc.}\\
       \affaddr{Mountain View, CA, USA}\\
       \email{\{clchen|raman|dtseng\}@google.com}
}


\maketitle 
\begin{abstract} 
  As the traditional methods of publishing and teaching in the STEM subjects are
  being progressively replaced by the provision of web-based content, ensuring
  full accessibility to this material for visually impaired users is of
  paramount importance for inclusive education. Already the text to speech
  translation of content for which well defined markup languages exists such as
  mathematics is a non-trivial problem. In this paper we present our efforts of
  making the speech translation of mathematical formulas a first class citizen
  in a general screen reader. We describe the ChromeVox screen reader and
  demonstrate how its features enable us to provide text to speech translation
  for web-based mathematical content. These features allow us to translate
  formulas given in a variety of formats into uniform speech strings exploiting
  ChromeVox's ability to handle alternative representations of DOM
  elements. This also allows us to customise aural rendering of mathematics both
  by using semantically enriched representations and employing flexible and
  adaptable speech rules. To further aid understanding of the math we exploit
  ChromeVox's idea of letting users engage with content on different levels of
  granularity to enable interactive exploration of complex mathematical
  formulas.
\end{abstract}
\category{H.5.2}{User Interfaces}{Voice I/O}
%A category including the fourth, optional field follows...
\category{H.5.4}{Hypertext}{Navigation, User Issues}

% \terms{}

\keywords{scree reader, mathematics, ChromeVox}

\section{Introduction}\label{sec:intro} As we move away from the traditional
methods of publishing and teaching to the provision of web-based content, more
and more scientific literature and teaching material in the STEM subjects
becomes available online. This material can range from traditional articles
containing mathematical formulas and scientific diagrams to highly interactive
webpages often exploiting novel media formats such as dynamic diagrams or
simulations. Ensuring full accessibility to this material for users that rely
primarily on voice output from their computer becomes a challenging
problem. Already the text to speech translation of fairly conventional
scientific content like mathematical formulae, which contain rich structure and
for which a well defined markup language exists, is non-trivial.


\ednote{Some related work here.}

In this paper we report on our effort to integrate the basic support
for voicing mathematical content on the web into the ChromeVox screen
reader. 


% A major prerequisite for the comprehension of scientific literature is for a
% reader to be able to engage with the content.

% this is not as straight forward for a 





We shall present some of the main challenges to making maths
on the web accessible, where it is given in a variety of ways and
formats. We shall outline the rule based approach we have pursued and
how it fits with ChromeVox's philosophy to enable users to explore
content at different levels of granularity. It has led to an
implementation of a flexible speech rule engine that allows to
customise the reading experience along several axes and that also
provides an API for easy adaptation to specialised content by users
and web site authors.

\section{Mathematics on the Web}
\label{sec:math}

Ideally, mathematical formulas on a webpage should be represented in its own
specialised markup language, MathML~\cite{mathml}. But although MathML is
officially part of the HTML5 standard~\cite{html5}, not all of the major
browsers also implement MathML rendering, hence support for formulas included in
pure MathML on web pages is sketchy. Consequently in reality mathematics on the
web comes in a variety of flavours. One can identify predominant three ways in which 
mathematical content is currently given on the web:
\begin{enumerate}
\item Pure MathML markup: this relies on the user viewing the
  page with a browser that renders MathML.
\item Rendered with MathJax: the webpage author ensures that mathematical
  content is rendered client-side independent of the user's browser by including
  the third party MathJax library in the page. Formulas can be given in several
  different markup languages.  and relying on the third party that renders
  mathematical content client side.
\item Pre-rendered images: Content is ensured to display correctly, by including
  images of formulas in the webpage. The original markup from which the content
  was originally rendered is often given in an attribute of the image tag.
\end{enumerate}

In order to enable access to the majority of mathematics that can be found on
the web today, one hasd to provide text-to-speech support for all of the above
formats, which we shall briefly sketch the remainder of this section.

\subsection{MathML}
\label{sec:mathml}

\edbf{briefly mention distinction between presentation and content MathML}

It provides markup for the most common layout elements encountered in
mathematics, such as sub- and superscripts, fractions, square roots etc.  In
addition it provides markup to define mathematics specific styles, spacing, etc.
as well as specialised attributes such as for fonts, accents, etc.


\subsection{MathJax Rendering}
\label{sec:mathjax}

MathJax is a javascript display engine~\cite{mathjax} with the goal to display
mathematical formulas consistently on all browsers. It can use a number of input
formats and translates them into simple HTML markup that visually render a maths
expression in a browser, regardless of whether the browser supports MathML.
In particular, it can work with MathML markup, {\LaTeX}, and
AsciiMath~\cite{asciimath}.



\subsection{Hidden Markup}\label{sec:images}

Traditionally mathematical formulas would be embedded into web content using
images generated from documents that were previously generated with systems
specialising on rendering mathematics, for example, {\LaTeX}. And despite the
obvious drawbacks of having content in images, such as lack of scalability, this
method is still an option that is chosen by many major web sites containing
mathematical content (e.g., Wikipedia~\cite{wikipedia},
MathWorld~\cite{mathworld}), to provide content consistently at reliable
performance, independent of third party libraries.  While the mathematics given
in image form is effectively inaccessible, the original source from which a
formula has been generated is often given in form of alternative text attribute
of the image tag.  For instance, in the case of Wikipedia, most of the
mathematics images have alternative text provided in \LaTeX, whereas Mathworld
contains AsciiMath markup. This form of hidden markup can then be displayed in
text-only browers or used for copy-and-paste operations,

In ChromeVox we exploit hidden markup as well, to ensure a consistent
screenreading experience of all mathematical content on the web.  In particular,
we rely on the MathJax library to translate hidden markup into a MathML
representation that we can speak instead. We describe in detail the treatment of
alternative representations in Sec.~\ref{sec:alternative}.

\begin{figure*}[t!]
  \begin{center}
\leavevmode

\lstxml
\begin{lstlisting}
<math xmlns="http://www.w3.org/1998/Math/MathML">
  <mstyle displaystyle="true">
    <mi>x</mi>
    <mo>=</mo>
    <mfrac>
      <mrow>
        <mo>&#x2212;<!-- \u2212 --></mo>
        <mi>b</mi>
        <mo>&#x00B1;<!-- � --></mo>
        <msqrt>
          <msup>
            <mi>b</mi>
            <mn>2</mn>
          </msup>
          <mo>&#x2212;<!-- \u2212 --></mo>
          <mn>4</mn>
          <mi>a</mi>
          <mi>c</mi>
        </msqrt>
      </mrow>
      <mrow>
        <mn>2</mn>
        <mi>a</mi>
      </mrow>
    </mfrac>
  </mstyle>
</math>

<todo> ... Mathjax <todo>


<img class="tex" 
      alt="x=\frac{-b \pm \sqrt {b^2-4ac}}{2a}" 
      src="//upload.wikimedia.org/math/3/c/a/3ca857f705daba6b9e6e6d3ccad7990f.png" />

\end{lstlisting}
\caption{Three representations of the quadratic formula.}
\label{fig:quadratic}
  \end{center}
\end{figure*}

\section{ChromeVox}
\label{sec:chromevox}



\begin{figure}[h!]
  \begin{center}
    \leavevmode
    \includegraphics[width=.4\columnwidth]{images/granularity1}
    \includegraphics[width=.4\columnwidth]{images/granularity2}
    \caption{Schematic depiction of granularities.}
    \label{fig:granularity}
  \end{center}
\end{figure}

\begin{figure}[h!]
  \begin{center}
    \leavevmode
    \includegraphics[width=.4\columnwidth]{images/walker1}
    \includegraphics[width=.4\columnwidth]{images/walker2}
    \caption{Schematic depiction of different walkers.}
    \label{fig:walkers}
  \end{center}
\end{figure}

\begin{figure}[h!]
  \begin{center}
    \leavevmode
    \includegraphics[width=.5\columnwidth]{images/substructure1}
    \caption{Schematic depiction of alternative represenatations.}
    \label{fig:walkers}
  \end{center}
\end{figure}

\section{Translating Mathematics}
\label{sec:translate}

\section{Exploring Mathematics}
\label{sec:explore}

\section{Alternative Representations}
\label{sec:alternative}




\section{Conclusions}
\label{sec:conc}


\bibliographystyle{plain} \bibliography{www_2014}

\end{document}
